\section{Basics}

\subsection{Separate Files}

Separating your chapters into separate files and \verb|\include|-ing them in the main file makes your \LaTeX\ code
cleaner. We create a section and subsections, and these are automatically populated in the table of contents with
clickable hyperlinks.

Note that the preamble was \verb|\input|ted rather than \verb|\include|d; you can read more about the differences
\href{https://tex.stackexchange.com/questions/246/when-should-i-use-input-vs-include}{\color{blue}{here}}.
% We have to specify the text colour here since it is modified elsewhere

\subsection{Some Typesetting}

\emph{Environments} provide certain functionality for typesetting. The syntax is as follows:

\begin{verbatim}
    \begin{environment}
        Content in the environment
    \end{environment}
\end{verbatim}

Itemized lists can be created as follows:

\begin{itemize}
    \item Some cool points
    \item Bullets can be customized using the \verb|enumitem| package
\end{itemize}

Additionally, numbered lists can be used.

\begin{enumerate}
    \item Another set of cool statements
    \item Numbering can also be customized using \verb|enumitem|
\end{enumerate}

You can use \verb|`| and \verb|'| for single opening and closing quotation marks (respectively), ``like this'' or
`this'.

To \emph{emphasize} certain text, use \verb|\emph{text}|. This is similar to \verb|\textbf{text}| (bold) and
\verb|\textit{text}| (italics), though \verb|emph| can be customized to use any specified font.

% TODO: add image example
