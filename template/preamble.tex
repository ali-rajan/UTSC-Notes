% LaTeX math notes template
% Author: Ali Rajan

\usepackage[letterpaper, margin=0.5in, top=0.25in, includehead]{geometry}
\usepackage{amsmath}                                    % Standard math package
\usepackage{amssymb}                                    % Math symbols
\usepackage{titlesec}                                   % Custom title formatting and options
\usepackage{parskip}                                    % Spacing options
\usepackage{tocloft}                                    % Table of contents formatting
\usepackage{setspace}                                   % Vertical spacing for the table of contents and fact list
\usepackage{enumitem}                                   % enumerate and itemize list formatting and options
\usepackage{fancyhdr}                                   % Headers and footers
\usepackage[linktoc=all]{hyperref}                      % Clickable hyperlinks for table of contents and theorems
\usepackage[nameinlink]{cleveref}                       % Advanced theorem referencing
\usepackage[amsmath, hyperref, thmmarks]{ntheorem}      % Enhanced theorem environments
\usepackage[most]{tcolorbox}                            % Fancy theorem boxes
% Packages used in the sections
\usepackage{commath}                                    % More math symbols/operators
\usepackage{pifont}                                     % Custom bullet symbols
% \usepackage{mathtools}                                % Additional
% \usepackage{graphicx}
\usepackage{xcolor}                                     % Setting text colour
\usepackage{etoolbox}                                   % For the \AtBeginEnvironment macro used

% General settings
\hypersetup{hidelinks}
\titleformat{\section}{\LARGE \bfseries}{Chapter \thesection}{1em}{}
\titleformat{\subsection}{\Large \bfseries}{\thesubsection}{1em}{}
\renewcommand{\cfttoctitlefont}{\LARGE \bfseries}
\titlespacing{\section}{0em}{10ex}{5ex}
\titlespacing{\subsection}{0em}{5ex}{3ex}
\allowdisplaybreaks[4]
\raggedbottom
\AtBeginEnvironment{bmatrix}{\everymath{\displaystyle}}
\AtBeginEnvironment{vmatrix}{\everymath{\displaystyle}}
\renewcommand{\arraystretch}{2}

\renewcommand{\cftsecfont}{\bfseries \Large}
\renewcommand{\cftsubsecfont}{\large}
\makeatletter
\renewcommand*\l@tcolorbox{\@dottedtocline{1}{1.5em}{4em}}  % Indendation options for tcblistof
\preto\env@matrix{\renewcommand{\arraystretch}{2}}  % Vertical stretch for matrices only
\makeatother

\renewcommand*{\labelitemi}{\ding{118}}
\renewcommand*{\labelitemiii}{\textbullet}

% Header options
\fancyhead[LEO]{Chapter \nouppercase \leftmark}     % Your top-left header text and section numbering here
\fancyhead[CO]{ABC~123}                             % Your top-center header text for odd pages only
% ~ is the same as a regular space, but it ensures that ABC and 123 are not split on separate lines
\fancyhead[REO]{University Name}                    % Your top-right header text

% Custom colours
\definecolor{myblue}{RGB}{21, 101, 192}
\definecolor{myred}{RGB}{198, 40, 40}
\definecolor{mylightblue}{RGB}{80, 140, 208}
\definecolor{mylightred}{RGB}{212, 94, 94}

% Theorem-like environments
\tcbuselibrary{theorems}
% Performs \newtcbtheorem[#1]{#2}{#3}{options}{}, where options created a styled theorem box with label #2, whos
% colour depends on #4
\newcommand*{\definestyledtcbtheorem}[4]{
    \newtcbtheorem[#1]{#2}{#3}
    {
        label type=#2,
        label separator=,
        % colframe=#4,
        colback=#4!10,
        frame hidden,
        opacityback=0.4,
        fonttitle=\bfseries,
        separator sign=\ ---,
        enhanced,
        colbacktitle=#4,
        attach boxed title to top left={xshift=2mm, yshift=-\tcboxedtitleheight/2},
        top=\tcboxedtitleheight/2,
        boxed title style={colframe=#4!10},
        parbox=false
    }{}
}
\definestyledtcbtheorem{number within=section, list inside=facts}{definition}{Definition}{myblue}
\definestyledtcbtheorem{use counter from=definition, list inside=facts}{axiom}{Axiom}{mylightblue}
\definestyledtcbtheorem{use counter from=definition, list inside=facts}{lemma}{Lemma}{mylightred}
\definestyledtcbtheorem{use counter from=definition, list inside=facts}{theorem}{Theorem}{myred}
\definestyledtcbtheorem{use counter from=definition, list inside=facts}{corollary}{Corollary}{mylightred}
\newtcbtheorem[use counter from=definition]{example}{Example}
{
    label type=example,
    label separator=,
    boxrule=0pt,
    boxsep=0pt,
    breakable,
    enhanced jigsaw,
    opacityback=0,
    borderline west={2pt}{0pt}{gray},
    frame hidden,
    bottomtitle=1mm,
    theorem hanging indent=0pt,
    fonttitle=\bfseries,
    coltitle=black,
    colbacktitle=white,
    attach title to upper,%
    separator sign none,
    description delimiters parenthesis,
    description font=\itshape\mdseries,
    after title=\textmd{:}\ ,%
    % description font=\mdseries,
    parbox=false
}{}
\newtcbtheorem[no counter]{remark}{Remark:}{
    boxrule=0pt,
    boxsep=0pt,
    breakable,
    enhanced jigsaw,
    opacityback=0,
    frame hidden,
    borderline horizontal={2pt}{0pt}{myred!25},
    attach title to upper,
    fonttitle=\bfseries,
    coltitle=myred,
    parbox=false
}{}
\theoremstyle{nonumberplain}
\theoremheaderfont{\scshape}
\theoremseparator{:}
\theorembodyfont{\upshape}
\theoremsymbol{ \ensuremath{\blacksquare}}  % The space ensures that the square isn't too close to any preceding text
\theorempreskip{2mm}
\newtheorem{proof}{Proof}
\crefname{enumi}{}{}
\creflabelformat{enumi}{#2(#1)#3}

% Environment for questions and answers
\newenvironment{questionparts}{\begin{enumerate}[label=(\alph*)]}{\end{enumerate}}
\newenvironment{solutionparts}
{\begin{enumerate}[label=(\alph*)] \setlength{\leftskip}{-0.4\leftmargin} \setlength{\rightskip}{-0.4\rightmargin}}
        {\end{enumerate}}

% Commands for math symbols and other macros
\newcommand*{\sect}{\S}
\newcommand*{\sects}{\S\S}
\newcommand*{\pg}{p.~}
\newcommand*{\pgs}{pp.~}
\newcommand*{\q}{Q}
\newcommand*{\aiff}{\; \iff & \;}   % Used in alignat lines of the form \iff && A & = B for proper spacing
\newcommand*{\aimplies}{\; \implies & \;}   % \implies analog for \aiff

\newcommand*{\vect}[1]{\mathbf{#1}}
\newcommand*{\field}{\mathbb{F}}
\newcommand*{\naturals}{\mathbb{N}}
\newcommand*{\integers}{\mathbb{Z}}
\newcommand*{\reals}{\mathbb{R}}
\newcommand*{\complex}{\mathbb{C}}
\newcommand*{\spn}{\operatorname*{span}}
\newcommand*{\kernel}{\operatorname*{ker}}
\newcommand*{\im}{\operatorname*{image}}
\newcommand*{\trace}{\operatorname*{trace}}

\newcommand*{\realmatrix}[2]{\mathbf{M}_{#1 \times #2} (\reals)}
\newcommand*{\realpoly}[1]{\mathbf{P}_{#1} (\reals)}
\newcommand*{\basis}[2][\beta]{\{#1_1, \ldots, #1_#2\}}
\newcommand*{\innerprod}[2]{\langle #1 \mid #2 \rangle}

% WARNING: exponents occurring right these parentheses/brackets this do not vertically align properly
\newcommand*{\p}[1]{\mathopen{}\left(#1\right)\mathclose{}}
\newcommand*{\intervalc}[1]{\left[#1\right]}
\newcommand*{\intervalo}[1]{\left(#1\right)}
\newcommand*{\intervalco}[1]{\left[#1\right)}
\newcommand*{\intervaloc}[1]{\left(#1\right]}

\newcommand*{\centeredimage}[2][0.5]{\begin{center}
        \includegraphics[scale=#1]{#2}
    \end{center}}
